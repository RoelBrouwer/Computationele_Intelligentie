\documentclass[]{report}
\usepackage[margin=1.0in]{geometry}


% Title Page
\title{Practicumverslag Computationele Intelligentie}
\author{Cornelis Bouter (studentnummer) en Roel Brouwer (3976866)}
\date{27 juni 2014}

% Code Snippet settings
\usepackage{listings}
\usepackage{color}

\usepackage[title,titletoc,toc]{appendix}

\usepackage{fancyhdr}

\renewcommand{\chaptername}{Hoofdstuk}

\definecolor{dkgreen}{rgb}{0,0.6,0}
\definecolor{gray}{rgb}{0.5,0.5,0.5}
\definecolor{mauve}{rgb}{0.58,0,0.82}

\lstset{
  frame=tb,
  language=Java,
  aboveskip=3mm,
  belowskip=3mm,
  showstringspaces=false,
  columns=flexible,
  basicstyle={\small\ttfamily},
  numbers=left,
  numberstyle=\tiny\color{gray},
  keywordstyle=\color{blue},
  commentstyle=\color{dkgreen},
  stringstyle=\color{mauve},
  breaklines=true,
  breakatwhitespace=true,
  tabsize=4
}

\begin{document}
\maketitle

\chapter{Implementatie algoritmen}
\section{Hill-climbing}
\subsection{Algemeen}
\subsubsection{Representatie en bepalen begintoestand}
Sudoku's zijn gerepresenteerd als een dubbel array ($n^2 \times n^2$)  van cijfers (1 t/m $n^2$). In dit hele verslag wordt met $n$ de wortel van de breedte (of hoogte) van de puzzel bedoeld. Voor een puzzel van 9 bij 9 geldt dus, bijvoorbeeld $n = 3$. We beschouwen dus alleen vierkante sudoku's. Probleemtoestanden zijn volledig ingevulde sudoku's, wat betekent dat er nog iets met de input moet gebeuren om een begintoestand te verkrijgen.

De lege vakjes in de sudoku worden gevuld door een willekeurig getal dat nog niet in het $n \times n$ blok voorkomt. Welk getal in het vakje terecht komt wordt bepaald door de volgende functie:

\begin{minipage}{\textwidth}
\begin{lstlisting}
ZoekVulling(AlGevuld)
	// AlGevuld is een boolean array, met voor elk getal dat
	// in een n bij n blok moet voorkomen een positie.
	// AantalOngevuld is een variabele waarin het aantal nog
	// ongevulde vakjes is opgeslagen.
	TeVullen = (int)(randomFloat(0, 1) * AantalOngevuld);
	count = 0;
	// Tot is exclusief.
	for i = 0 tot n * n
		if (!AlGevuld[i])
			if(count == TeVullen)
				return i + 1;
			else 
				count++;
\end{lstlisting}
\end{minipage}

Het resultaat is een volledig gevulde begintoestand.

\subsubsection{Evaluatie}
Het is nodig om toestanden te beoordelen, zodat ze vergeleken kunnen worden en we daadwerkelijk kunnen klimmen.

De onderstaande evaluatiefunctie doet dat door het aantal missende cijfers in alle rijen en kolommen bij elkaar op te tellen:

\begin{minipage}{\textwidth}
\begin{lstlisting}
Evalueer (Sudoku)
	TeReturnen = 0;
	contains = boolean[n * n];
	// Telt alle ontbrekende cijfers in de rijen op.
	for i = 0 tot n * n
		for k = 0 tot n * n 
			contains[k] = false;
		for j = 0 tot n * n
			contains[Sudoku[i][j] - 1] = true;
		for k = 0 tot n * n
			if(!contains[k]) 
				TeReturnen++;
	// Telt alle ontbrekende cijfers in de kolommen op.
	for i = 0 tot n * n
		for k = 0 tot n * n
			contains[k] = false;
		for j = 0 tot n * n
			contains[Sudoku[j][i] - 1] = true;
		for k = 0 tot n * n
			if(!contains[k]) 
				TeReturnen++;
	return TeReturnen;
\end{lstlisting}
\end{minipage}

\subsubsection{Zoekoperator}
Het bepalen van de volgende toestand in het \textit{hill-climbing} gebeurt aan de hand van de zoekoperator (hieronder in pseudo-code uitgewerkt). Deze geeft een nieuwe sudoku terug, die van de oude verschilt op twee vakjes: de eerste verwisseling die een verbetering opleverde in de evaluatiewaarde van de sudoku.

\begin{minipage}{\textwidth}
\begin{lstlisting}
ZoekOperator (Sudoku)
	// eval bevat de waarde die de evaluatiefunctie aan de sudoku geeft.
	eval = Sudoku.evalueer();
	// Verwisselen gaat per nxn-blok. g en h representeren het blok. Het zijn constantes die bij 
	// de variabelen die de positie van de vakjes binnen het blok representeren (i, j, k en l) 
	// worden opgeteld om het juiste vakje te kunnen identificeren. Tot is exclusief. In de 
	// volgende geneste loops worden alle wissel-mogelijkheden afgelopen tot er een gevonden is 
	// die de huidige toestand verbetert.
	for g = 0 tot n*n (ophogen per 3) 
		for h = 0 tot n*n (ophogen per 3)
			for i = 0 tot n 
				for j = 0 tot n
					for k = 0 tot n
						for l = 0 tot n
							if ((i,j) en (k,l) representeren niet hetzelfde vakje EN
							    (i,j) en (k,l) zijn variabele vakjes)
								// Verkrijg de nieuwe sudoku door de vakjes te verwisselen.
								Sudoku.verwissel(g + i, h + j, g + k, h + l);
								// Als de nieuwe beter is zijn we klaar.
								if (Sudoku.evalueer() > eval) 
									return Sudoku;
								// Anders maken we de verwisseling ongedaan.
								else
									Sudoku.verwissel(g + i, h + j, g + k, h + l);
	// Als er niks verwisseld kan worden, zitten we in een lokaal
	// optimum. We geven niets terug.
	return null;
\end{lstlisting}
\end{minipage}

Een lokaal optimum (of plateau) is gevonden op het moment dat de zoekoperator niets teruggeeft: alle successor-toestanden zijn dan immers slechter dan of even goed als de huidige toestand.

Het is noodzakelijk dat deze zoekoperator-functie systematisch alle mogelijk successors langsgaat. Op die manier kunnen we er zeker van zijn dat geen van de successor-toestanden meer een verbetering is, en weten we dat we in een lokaal optimum (of plateau) zitten. 

\subsection{Random Restart}
Met het klaarstaan van de zoekoperator die in de vorige sectie besproken is, is het implementeren van \textit{random restart} relatief eenvoudig. In pseudo-code ziet dat er als volgt uit:

\begin{minipage}{\textwidth}
\begin{lstlisting}
RandomRestart (HuidigeSudoku, Input)
	// Input is de originele input waaruit de begintoestand gehaald is.
	// Totdat een globaal optimum is gevonden, blijven we zoeken.
	while (Evalueer(HuidigeSudoku) > 0)
		NieuweSudoku = ZoekOperator(HuidigeSudoku);
		// Als een lokaal optimum wordt bereikt, herstarten.
		if (NieuweSudoku == null)
			// Maak een nieuwe begintoestand en begin opnieuw.
			HuidigeSudoku = Sudoku(n, Input);
		else
			// Anders gaan we door met de gevonden toestand. 
			HuidigeSudoku = NieuweSudoku;
	// Een globaal optimum is gevonden.
	Print(HuidigeSudoku);
\end{lstlisting}
\end{minipage}

\subsection{Iterated Local Search}
\subsubsection{Zoekoperator}
Voor het implementeren van \textit{iterated local search} is het nodig geweest de zoekoperator iets aan te passen, of beter gezegd: een variant op de zoekfunctie toe te voegen. 

De zoekoperator die voor het beklimmen van de heuvel, het \textit{hill-climbing} deel, gebruikt wordt, loopt alle swap-mogelijkheden in een bepaalde volgorde door, en kiest de eerste die een verbetering is. Voor \textit{iterated local search} is een \textit{random walk} nodig. Een \textit{random walk} geeft niets om het antwoord op de vraag of een opvolger-toestand al dan niet een verbetering is. Het moet een willekeurige opvolger-toestand kunnen kiezen, ongeacht de waarde die de evaluatiefunctie aan die toestand geeft. 

Wanneer de oude implementatie gebruikt zou worden, zou het dan steeds voor de eerste optie kiezen: het swappen van de eerste twee niet-vaste vakjes in het blok in de linksbovenhoek van de sudoku. Dat willen we niet, daarom veranderen we de zoekoperator-functie zodat deze twee willekeurige vakjes (binnen hetzelfde blok - natuurlijk) van plaats verwisseld worden:

\begin{minipage}{\textwidth}
\begin{lstlisting}
RandomZoekOperator (Sudoku)
	// g en h representeren het blok waarin het verwisselen wordt uitgevoerd.
	g, h = randomInt(0, n) * n;
	// i, j, k en l representeren de twee vakjes die verwisseld worden.
	i, j, k, l = randomInt(0, n);
	while ((i,j) en (k,l) representeren hetzelfde vakje OF
	  (i,j) of (k,l) is een vast vakje)
		// Nieuwe waardes voor i, j, k en l.
		i, j, k, l = randomInt(0, n);
	// Verwissel de genoemde vakjes in de sudoku en return het resultaat
	return Sudoku.verwissel(g + i, h + j, g + k, h + l);
\end{lstlisting}
\end{minipage}

\subsubsection{Zoek algoritme}
Gebruik makend van de hierboven beschreven zoekoperator(s) (zowel de willekeurige als de deterministische versie), is de implementatie van \textit{iterated local search} als volgt:

\begin{minipage}{\textwidth}
\begin{lstlisting}
IteratedLocalSearch (HuidigeSudoku)
	// Totdat een globaal optimum is gevonden, blijven we zoeken.
	while (Evalueer(HuidigeSudoku) > 0)
		NieuweSudoku = ZoekOperator(HuidigeSudoku);
		// Als een lokaal optimum wordt bereikt, daar weglopen.
		if (NieuweSudoku == null)
			// Maak een random walk van S lang, S kan gevarieerd worden.
			for i = 0 tot S
				HuidigeSudoku = RandomZoekOperator(HuidigeSudoku);
		else 
			// Anders gaan we door met de gevonden toestand.
			HuidigeSudoku = NieuweSudoku;
	// Een globaal optimum is gevonden.
	Print(HuidigeSudoku);
\end{lstlisting}
\end{minipage}

\section{Constraint Satisfaction}
\subsection{Algemeen}
\subsubsection{Representatie}
Een \textit{Constraint Satisfaction Problem} (CSP) bestaat uit drie componenten: de variabelen ($X$), de domeinen ($D$) en de constraints ($C$). Hieronder beschrijven we hoe we elk van deze componenten hebben gerepresenteerd in onze code.
\begin{itemize}
\item \textbf{Variabelen ($X$)}. De variabelen zijn vakjes in de sudoku. Deze vakjes zijn gerepresenteerd als objecten met een waarde en een domein en methoden om informatie over deze twee zaken op te vragen of ze te manipuleren. "Vaste vakjes" zijn gerepresenteerd door vanaf het begin maar één element in het domein te stoppen.
\item \textbf{Domeinen ($D$)}. De domeinen zitten ingebakken in de objecten die hierboven beschreven zijn, aangezien de relatie variabele-domein in principe 1 op 1 is.
\item \textbf{Constraints ($C$)}. Constraints zijn ook gerepresenteerd als objecten. We hebben ervoor gekozen om alleen binaire constraints te implementeren, aangezien alle hogere constraints tot binaire constraints te herleiden zijn. Een constraint is een object dat twee vakjes "bij zich heeft" en methoden bevat om te kijken welke vakjes dat zijn en of het constraint geschonden is. Aangezien in dit probleem alleen constraints van de vorm $ X \not= Y $ voorkomen is in die laatste methode ingebakken dat getest wordt of de waarde van het ene vakje ongelijk is aan de waarde van het andere vakje.
\end{itemize}
Het probleem (de sudoku) is gerepresenteerd als een lijst van variabelen en een lijst van constraints.

Elke variabele heeft dan $2(n^2 - 1) + (n - 1)^2$ constraints. Er zitten $n^2 - 1$ andere vakjes in dezelfde rij, en evenveel in dezelfde kolom. Er zitten daarbij nog $n^2$ vakjes in hetzelfde blok, daarvan moeten we  het vakje zelf (1) en de vakjes die al in de rij/kolom zitten (2(n - 1)) nog afhalen. Dan komen we uit op $(n^2 - (1 + 2(n - 1))) = (n - 1)^2$ extra vakjes.

Hieronder laten we zien hoe het probleem gerepresenteerd wordt, door een (deel van de) constructor van het object dat het probleem representeert te tonen.

\begin{minipage}{\textwidth}
\begin{lstlisting}
Sudoku(Input)
	// Input is de originele input waaruit de begintoestand gehaald is.
	X = Variabelen[n * n][n * n];
	// l en k voor de blokken, i en j identificerren de vakjes in de blokken.
	for l = 0 tot n
		for k = 0 tot n
			for i = 0 tot n
				for j = 0 tot n
					waarde = Input[l * n + i][k * n + j];
					// Als het een leeg vakje is, maken we een variabelen aan met een volledig domein.
					if (waarde == 0)
						X[l * n + i][k * n + j] = Variabelen(0, VolDomein);
					// Anders is het een gevuld vakje, en bevat het domein maar een element.
					else
						X[l * n + i][k * n + j] = Variabelen(waarde, EnkelDomein);
						
	Constraints = Constraint[n * n][n * n][2 * (n * n - 1) + ((n - 1) * (n - 1))];
	// Er worden constraints aangemaakt voor alle mogelijke combinaties van twee vakjes die in 
	// dezelfde rij, kolom, of hetzelfde blok zitten. Deze worden bij de juiste vakjes in het 
	// array gestopt, zodat aan het eind van het verhaal elk blokje een lijst heeft van de 
	// constraints die voor hem gelden. De precieze uitwerking daarvan is voor dit verslag 
	// niet relevant en kan in de code worden teruggevonden.
\end{lstlisting}
\end{minipage}

\subsubsection{Consistency check}
Het is belangrijk om na elke stap in het zoeken naar de oplossing na te gaan of met het nemen van deze stap geen constraints geschonden worden.

Gegeven dat een bepaald vakje gewijzigd is, wordt met behulp van onderstaande methode gecontroleerd of er geen constraints geschonden worden:

\begin{minipage}{\textwidth}
\begin{lstlisting}
Consistent(X, Y)
	// X en Y zijn de "coördinaten" van het betreffende vakje in de array.
	for i = 0 tot Constraints[X][Y].length
		// Als een van de constraints geschonden wordt, moet de methode false teruggeven.
		if (Constraints[X][Y][i].ConstraintGeschonden())
			return false;
	return true;
\end{lstlisting}
\end{minipage}

\subsection{Backtracking}
Gegeven de hierboven beschreven representatie van CSP is het relatief eenvoudig om backtracking te implementeren. Dit gebeurt recursief, waarbij steeds vakjes worden ingevuld, tot het mis gaat. Op dat moment wordt de laatste keuze herzien, totdat alle opties "op" zijn, of er een oplossing gevonden is. De onderstaande pseudocode laat zien hoe wij dit ge\"{i}mplementeerd hebben. 

Er wordt aan enkele hulpmethoden gerefereerd, waarvan geen pseudocode beschikbaar is in dit verslag. Dit zijn vrij eenvoudige methoden, waarvan de implementatie voor zich spreekt. Voor de precieze implementatie verwijzen we naar de code.

\begin{minipage}{\textwidth}
\begin{lstlisting}
Backtracking(Sudoku)
	NieuweSudoku = BacktrackingRecursief(Sudoku);
	if (NieuweSudoku != null)
		Print(NieuweSudoku);
	else
		Print("Geen oplossing.");

BacktrackingRecursief(Sudoku)
	// Als de sudoku volledig ingevuld is, zijn we klaar in de recursie
	if(Sudoku.volledigIngevuld())
		return Sudoku;
	
	// We pakken het eerste vakje (variabele) dat nog niet is ingevuld.	
	Vakje = VindVolgende(Sudoku);
	
	for i = 0 tot Vakje.domein.length
		// Als i in het domein zit, gaan we eens proberen om Vakje in te vullen met i + 1. We zetten de waarde op i + 1 (en verkleinen ook het domein zodat het alleen nog i + 1 bevat).
		if (Vakje.ElementInDomein(i + 1)) {
			OudDomein = Vakje.domein;
			Vakje.waarde = i + 1;
			// Als er geen constraints geschonden worden, gaan we een niveau dieper de recursie in.
			if (Sudoku.Consistent(Vakje.X, Vakje.Y))
				NieuweSudoku = BacktrackingRecursief(Sudoku);
				if (NieuweSudoku != null) return NieuweSudoku;
			// Als we hier terecht komen is het blijkbaar mislukt en kan i + 1 niet in Vakje worden ingevuld. We maken het vakje dus weer leeg en zetten het domein netjes terug. 
			Vakje.waarde = 0;
			Vakje.domein = OudDomein;
	// Er kan niets voor dit vakje worden ingevuld, dus de puzzel is onoplosbaar:
	return null;
\end{lstlisting}
\end{minipage}

\subsection{Forward Checking}
TODO

\chapter{Experimentele resultaten}
\section{Hill-climbing}
\subsection{Algemene observaties}
\subsubsection{Effici\"{e}ntie}
De tijd die het algoritme nodig heeft om een lokaal maximum te bereiken laat zich opvallend goed middellen. De gemiddelden van het aantal stappen dat nodig is van 10-15 lokale maxima liggen heel dicht bij elkaar - zolang het om dezelfde puzzel gaat. Tussen de puzzels is meer variatie te zien. Het lijkt erop dat het landschap van bepaalde puzzels rijker bezaaid is met lokale maxima dan dat van andere.

Voor puzzels waar $n=3$ ligt dit aantal in onze tests meestal rond de 7, en komt vrijwel nooit boven de 30. 

Voor puzzels waar $n=4$ ligt het aantal stappen dat nodig is om een lokaal maximum te bereiken meestal tussen de 30 en de 40, en bij $n=5$ tussen de 70 en de 90.

Voor het vinden van een lokaal maximum is verwaarloosbaar veel tijd nodig. Als dat je doel is, krijg je bij het stellen van de vraag direct een antwoord. Wij zijn echter meer geïnteresseerd in globale maxima, oplossingen voor de puzzel. Om die te bereiken is meer nodig dan hill-climbing alleen. Hieronder bespreken we de effici\"{e}ntie van de twee mogelijkheden die wij ge\"{i}mplementeerd hebben.

\subsubsection{Accepteren gelijke toestanden}
Bij de implementatie van hill-climbing kan ervoor gekozen worden om bij het zoeken naar een opvolger-toestand toestanden die niet strikt een verbetering zijn ten opzichte van de huidige toestand te accepteren als opvolger. Dit kan bereikt worden door de $>$ in regel 20 van de pseudocode voor de ZoekOperator (p. 2) te veranderen in een $\geq$.

In onze implementatie is dat een groot nadeel, omdat daarmee de mogelijkheid wordt geschapen tot het oneindig in cirkeltjes lopen - dat wordt zelfs erg waarschijnlijk. Op het moment dat de eerste swap die een opvolgertoestand oplevert die gelijk/beter is dan de huidige toestand een toestand oplevert die even goed is als de huidige toestand, zal bij de volgende stap de omgekeerde operatie (het terugswappen van de net verwisselde vakjes) de eerste swap zijn die gelijk/beter is dan de toestand. Omdat de mogelijke verwisselingen steeds in dezelfde volgorde worden beschouwd, betekent dit een oneindig heen-en-weer springen tussen twee toestanden, zonder ooit bij een lokaal maximum aan te komen.

Wanneer we dus kiezen voor $>$ in plaats van $\geq$ hebben we meer kans op een (snel) antwoord. Het is dus effici\"{e}nter en veiliger om daarvoor te kiezen.

\subsection{Random Restart}
Voor een uitgebreid(er) overzicht van de test-resultaten, zie \textit{appendix A}.
\subsubsection{Effici\"{e}ntie}
(Rekentijd en aantal toestanden tot globaal optimum)
\subsection{Iterated Local Search}
Voor een uitgebreid(er) overzicht van de test-resultaten, zie \textit{appendix B}.
\subsubsection{Effici\"{e}ntie}
(Rekentijd en aantal toestanden tot globaal optimum)
\section{Constraint Satisfaction}
\subsection{Backtracking}
\subsection{Forward Checking}

\chapter{Reflectie}

\begin{appendices}
{\Huge \textbf{Appendices - introductie}}\newline

Voor het testen van de verschillende algoritmen hebben we vijf puzzels van elke soort ($n = 3, 4, 5$) gebruikt. Eén hiervan is afkomstig van de website van het vak, de andere vier komen van \textit{menneske.no}. De puzzels zijn te vinden in de txt-bestanden met de namen test\_n$x$\_$nr$\_$bron$.txt, waarbij $x \in \{3,4,5\}, nr \in \{1,2,3,4,5\}$ en $bron \in \{ci,menneske\}$.

De testresultaten zijn uitgesplitst naar het soort puzzel (de $n$-waarde) en de specifieke puzzel waarop getest is. In de paragraaf bij de puzzel wordt vervolgens precies aangegeven wat getest is.

\chapter{Testresultaten Random Restart Hill-climbing}
\section{9 * 9 puzzels (n = 3)}
Voor elke puzzel in deze sectie zijn 15 testruns uitgevoerd. Van deze runs wordt het volgende weergeven:
\begin{itemize}
\item Lokale maxima. Het algoritme bereikt eerst een aantal lokale maxima voor het globale maximum bereikt wordt. Van de eerste 15 lokale maxima die bereikt zijn wordt hier weergeven:
\begin{itemize}
\item Het gemiddeld aantal stappen dat het kostte om vanaf de begintoestand naar het lokale maximum te komen;
\item De gemiddelde waarde van de evaluatiefunctie voor deze lokale maxima (m.a.w. hoeveel cijfers staan er gemiddeld nog verkeerd?).
\end{itemize}
\item Globaal maximum. Uiteindelijke bereikt het algoritme meestal een globaal maximum (als dat niet binnen 15 minuten gebeurd is, bevatten de kolommen over het globale maximum geen informatie). Van het globale maximum wordt weergeven:
\begin{itemize}
\item Het aantal stappen dat gezet is voor het globale maximum bereikt was;
\item De tijd (in milliseconden) die het kostte het globale maximum te bereiken;
\item Hoeveel lokale maxima langsgekomen zijn voordat het globale maximum gevonden was (= hoeveel keer een random restart heeft plaatsgevonden);
\item Hoeveel stappen er vanaf de laatste herstart gezet zijn om het globale maximum te bereiken.
\end{itemize}
\end{itemize}
\subsection{Puzzel 1 (test\_n3\_1\_ci)}
\begin{tabular}{|c||c|c||c|c|c|c|}
\hline  & \multicolumn{2}{|c||}{Lokale maxima} & \multicolumn{4}{|c|}{Globaal maximum}  \\ 
\hline Testnr. & stappen & evaluatie & stappen & tijd (ms) & lokale optima & laatste klim \\ 
\hline \hline 1 & 24 & 12 & 83801 & 12661 & 3243 & 31 \\ 
\hline 2 & 24 & 9,9 & 28602 & 4376 & 1114 & 31 \\ 
\hline 3 & 25 & 8,9 & 79208 & 11866 & 3085 & 44 \\ 
\hline 4 & 24 & 9,3 & 88237 & 13550 & 3447 & 31 \\ 
\hline 5 & 25 & 10,9 & 236198 & 35407 & 9245 & 33 \\ 
\hline	6	&	26	&	8,8	&	203321	&	30898	&	7909	&	31	\\
\hline	7	&	26	&	10,2	&	74913	&	11450	&	2920	&	32	\\
\hline	8	&	26	&	11,1	&	25339	&	3873	&	975	&	26	\\
\hline	9	&	24	&	9,7	&	168835	&	25533	&	6597	&	29	\\
\hline	10	&	24	&	11,2	&	24687	&	3768	&	954	&	21	\\
\hline	11	&	26	&	10,3	&	63134	&	9521	&	2456	&	26	\\
\hline	12	&	26	&	9,5	&	218689	&	32873	&	8529	&	28	\\
\hline	13	&	26	&	9,8	&	87454	&	13186	&	3405	&	30	\\
\hline	14	&	25	&	10,2	&	126969	&	18006	&	4947	&	25	\\
\hline	15	&	25	&	10,2	&	51102	&	7718	&	1977	&	37	\\
\hline \hline Gemiddeld: & 25,1 & 10,1 & 104032,6 & 15645,7 & 4053,5 & 30,3 \\ 
\hline 
\end{tabular} 
\subsection{Puzzel 2 (test\_n3\_2\_menneske)}
\begin{tabular}{|c||c|c||c|c|c|c|}
\hline  & \multicolumn{2}{|c||}{Lokale maxima} & \multicolumn{4}{|c|}{Globaal maximum}  \\ 
\hline Testnr. & stappen & evaluatie & stappen & tijd (ms) & lokale optima & laatste klim \\ 
\hline \hline	1	&	8	&	15	&	\multicolumn{4}{|c|}{\textit{Werd niet binnen 15 minuten gevonden}}\\
\hline	2	&	8	&	16	&		&		&		& \\
\hline	3	&	7	&	15	&		&		&		& \\
\hline	4	&	10	&	15	&		&		&		& \\
\hline	5	&	7	&	16	&		&		&		& \\
\hline	6	&	8	&	15	&		&		&		& \\
\hline	7	&	6	&	17	&		&		&		& \\
\hline	8	&	7	&	16	&		&		&		& \\
\hline	9	&	8	&	15	&		&		&		& \\
\hline	10	&	7	&	18	&		&		&		& \\
\hline	11	&	8	&	15	&		&		&		& \\
\hline	12	&	7	&	17	&		&		&		& \\
\hline	13	&	8	&	15	&		&		&		& \\
\hline	14	&	8	&	16	&		&		&		& \\
\hline	15	&	8	&	15	&		&		&		& \\

\hline \hline Gemiddeld: & 7,7 & 15,7 &  &  &  &  \\ 
\hline 
\end{tabular}
\subsection{Puzzel 3 (test\_n3\_3\_menneske)}
\begin{tabular}{|c||c|c||c|c|c|c|}
\hline  & \multicolumn{2}{|c||}{Lokale maxima} & \multicolumn{4}{|c|}{Globaal maximum}  \\ 
\hline Testnr. & stappen & evaluatie & stappen & tijd (ms) & lokale optima & laatste klim \\ 
\hline \hline	1	&	7	&	17	&	\multicolumn{4}{|c|}{\textit{Werd niet binnen 15 minuten gevonden}}	\\
\hline	2	&	7	&	17	&		&		&		&	\\
\hline	3	&	7	&	17	&		&		&		&	\\
\hline	4	&	7	&	18	&		&		&		&	\\
\hline	5	&	8	&	16	&		&		&		&	\\
\hline	6	&	7	&	16	&		&		&		&	\\
\hline	7	&	7	&	16	&		&		&		&	\\
\hline	8	&	7	&	16	&		&		&		&	\\
\hline	9	&	9	&	17	&		&		&		&	\\
\hline	10	&	8	&	17	&		&		&		&	\\
\hline	11	&	7	&	17	&		&		&		&	\\
\hline	12	&	8	&	17	&		&		&		&	\\
\hline	13	&	7	&	18	&		&		&		&	\\
\hline	14	&	7	&	17	&		&		&		&	\\
\hline	15	&	7	&	17	&		&		&		&	\\

\hline \hline Gemiddeld: & 7,3 & 16,9 &  &  &  &  \\ 
\hline 
\end{tabular}
\subsection{Puzzel 4 (test\_n3\_4\_menneske)}
\begin{tabular}{|c||c|c||c|c|c|c|}
\hline  & \multicolumn{2}{|c||}{Lokale maxima} & \multicolumn{4}{|c|}{Globaal maximum}  \\ 
\hline Testnr. & stappen & evaluatie & stappen & tijd (ms) & lokale optima & laatste klim \\ 
\hline \hline	1	&	7	&	16	&	\multicolumn{4}{|c|}{\textit{Werd niet binnen 15 minuten gevonden}}	\\
\hline	2	&	7	&	14	&		&		&		&	\\
\hline	3	&	7	&	15	&		&		&		&	\\
\hline	4	&	6	&	16	&		&		&		&	\\
\hline	5	&	8	&	15	&		&		&		&	\\
\hline	6	&	7	&	15	&		&		&		&	\\
\hline	7	&	7	&	14	&		&		&		&	\\
\hline	8	&	7	&	16	&		&		&		&	\\
\hline	9	&	7	&	16	&		&		&		&	\\
\hline	10	&	8	&	15	&		&		&		&	\\
\hline	11	&	7	&	15	&		&		&		&	\\
\hline	12	&	7	&	15	&		&		&		&	\\
\hline	13	&	6	&	16	&		&		&		&	\\
\hline	14	&	7	&	15	&		&		&		&	\\
\hline	15	&	7	&	16	&		&		&		&	\\
\hline \hline Gemiddeld: & 7 & 15,3 &  &  &  &  \\ 
\hline 
\end{tabular}
\subsection{Puzzel 5 (test\_n3\_5\_menneske)}
\begin{tabular}{|c||c|c||c|c|c|c|}
\hline  & \multicolumn{2}{|c||}{Lokale maxima} & \multicolumn{4}{|c|}{Globaal maximum}  \\ 
\hline Testnr. & stappen & evaluatie & stappen & tijd (ms) & lokale optima & laatste klim \\ 
\hline \hline	1	&	6	&	16	&	2017	&	280	&	86	&	32	\\
\hline	2	&	7	&	15	&	10128	&	1235	&	433	&	32	\\
\hline	3	&	6	&	14	&	13049	&	1566	&	555	&	34	\\
\hline	4	&	7	&	16	&	60000	&	6989	&	2566	&	32	\\
\hline	5	&	7	&	15	&	11866	&	1406	&	509	&	35	\\
\hline	6	&	6	&	16	&	60588	&	7048	&	2602	&	25	\\
\hline	7	&	6	&	14	&	9328	&	1146	&	391	&	27	\\
\hline	8	&	7	&	15	&	5366	&	673	&	226	&	28	\\
\hline	9	&	6	&	18	&	9808	&	1184	&	412	&	28	\\
\hline	10	&	6	&	16	&	49993	&	5839	&	2146	&	34	\\
\hline	11	&	5	&	16	&	15863	&	1933	&	6683	&	27	\\
\hline	12	&	5	&	16	&	60006	&	6895	&	2564	&	30	\\
\hline	13	&	6	&	17	&	64164	&	7435	&	2753	&	29	\\
\hline	14	&	6	&	15	&	33467	&	3862	&	1421	&	32	\\
\hline	15	&	6	&	14	&	83975	&	9668	&	3592	&	37	\\
\hline \hline Gemiddeld: & 6,1 & 15,5 & 32641,2 & 3810,6 & 1795,9 & 30,8 \\ 
\hline 
\end{tabular}
\section{16 * 16 puzzels (n = 4)}
Voor elke puzzel in deze sectie zijn 5 testruns uitgevoerd. Bij geen enkele van de puzzels werd binnen 15 minuten een oplossing bereikt. Er wordt dus alleen maar data weergegeven van de eerste 10 gevonden lokale maxima:
\begin{itemize}
\item Het gemiddeld aantal stappen dat het kostte om vanaf de begintoestand naar het lokale maximum te komen;
\item De gemiddelde waarde van de evaluatiefunctie voor deze lokale maxima (m.a.w. hoeveel cijfers staan er gemiddeld nog verkeerd?).
\end{itemize}
\subsection{Puzzel 1 (test\_n4\_1\_ci)}
\begin{tabular}{|c||c|c|}
\hline Testnr. & stappen & evaluatie \\ 
\hline \hline 1 & 31 & 101 \\ 
\hline 2 & 30 & 100 \\ 
\hline 3 & 32 & 99 \\ 
\hline 4 & 28 & 102 \\ 
\hline 5 & 29 & 101 \\ 
\hline \hline Gemiddeld: & 30 & 100,6 \\ 
\hline 
\end{tabular} 
\subsection{Puzzel 2 (test\_n4\_2\_menneske)}
\begin{tabular}{|c||c|c|}
\hline Testnr. & stappen & evaluatie \\ 
\hline \hline 1 & 35 & 83 \\ 
\hline 2 & 35 & 87 \\ 
\hline 3 & 35 & 89 \\ 
\hline 4 & 36 & 85 \\ 
\hline 5 & 38 & 86 \\ 
\hline \hline Gemiddeld: & 35,8 & 86 \\ 
\hline 
\end{tabular} 
\subsection{Puzzel 3 (test\_n4\_3\_menneske)}
\begin{tabular}{|c||c|c|}
\hline Testnr. & stappen & evaluatie \\ 
\hline \hline 1 & 36 & 89 \\ 
\hline 2 & 39 & 81 \\ 
\hline 3 & 38 & 84 \\ 
\hline 4 & 33 & 87 \\ 
\hline 5 & 35 & 89 \\ 
\hline \hline Gemiddeld: & 36,2 & 86 \\ 
\hline 
\end{tabular} 
\subsection{Puzzel 4 (test\_n4\_4\_menneske)}
\begin{tabular}{|c||c|c|}
\hline Testnr. & stappen & evaluatie \\ 
\hline \hline 1 & 36 & 95 \\ 
\hline 2 & 35 & 91 \\ 
\hline 3 & 36 & 89 \\ 
\hline 4 & 39 & 91 \\ 
\hline 5 & 36 & 89 \\ 
\hline \hline Gemiddeld: & 36,4 & 91 \\ 
\hline 
\end{tabular} 
\subsection{Puzzel 5 (test\_n4\_5\_menneske)}
\begin{tabular}{|c||c|c|}
\hline Testnr. & stappen & evaluatie \\ 
\hline \hline 1 & 29 & 91 \\ 
\hline 2 & 32 & 88 \\ 
\hline 3 & 32 & 87 \\ 
\hline 4 & 29 & 89 \\ 
\hline 5 & 33 & 87 \\ 
\hline \hline Gemiddeld: & 31 & 88,4 \\ 
\hline 
\end{tabular} 

\section{25 * 25 puzzels (n = 5)}
Voor de eerste twee puzzels in deze sectie zijn 5 testruns uitgevoerd, voor de laatste drie zijn dat er 3. Bij geen enkele van de puzzels werd binnen 15 minuten een oplossing bereikt. Er wordt dus alleen maar data weergegeven van de eerste 10 gevonden lokale maxima:
\begin{itemize}
\item Het gemiddeld aantal stappen dat het kostte om vanaf de begintoestand naar het lokale maximum te komen;
\item De gemiddelde waarde van de evaluatiefunctie voor deze lokale maxima (m.a.w. hoeveel cijfers staan er gemiddeld nog verkeerd?).
\end{itemize}
\subsection{Puzzel 1 (test\_n5\_1\_ci)}
\begin{tabular}{|c||c|c|}
\hline Testnr. & stappen & evaluatie \\ 
\hline \hline 1 & 72 & 269 \\ 
\hline 2 & 67 & 260 \\ 
\hline 3 & 70 & 270 \\ 
\hline 4 & 72 & 265 \\ 
\hline 5 & 69 & 268 \\ 
\hline \hline Gemiddeld: & 70 & 266,4 \\ 
\hline 
\end{tabular} 
\subsection{Puzzel 2 (test\_n5\_2\_menneske)}
\begin{tabular}{|c||c|c|}
\hline Testnr. & stappen & evaluatie \\ 
\hline \hline 1 & 82 & 210 \\ 
\hline 2 & 90 & 202 \\ 
\hline 3 & 83 & 212 \\ 
\hline 4 & 82 & 205 \\ 
\hline 5 & 87 & 199 \\ 
\hline \hline Gemiddeld: & 84,8 & 205,6 \\ 
\hline 
\end{tabular} 
\subsection{Puzzel 3 (test\_n5\_3\_menneske)}
\begin{tabular}{|c||c|c|}
\hline Testnr. & stappen & evaluatie \\ 
\hline \hline 1 & 87 & 219 \\ 
\hline 2 & 90 & 220 \\ 
\hline 3 & 83 & 227 \\ 
\hline \hline Gemiddeld: & 86,7 & 222 \\ 
\hline 
\end{tabular} 
\subsection{Puzzel 4 (test\_n5\_4\_menneske)}
\begin{tabular}{|c||c|c|}
\hline Testnr. & stappen & evaluatie \\ 
\hline \hline 1 & 85 & 215 \\ 
\hline 2 & 81 & 219 \\ 
\hline 3 & 85 & 220 \\ 
\hline \hline Gemiddeld: & 83,7 & 218 \\ 
\hline 
\end{tabular} 
\subsection{Puzzel 5 (test\_n5\_5\_menneske)}
\begin{tabular}{|c||c|c|}
\hline Testnr. & stappen & evaluatie \\ 
\hline \hline 1 & 83 & 226 \\ 
\hline 2 & 87 & 225 \\ 
\hline 3 & 84 & 220 \\ 
\hline \hline Gemiddeld: & 84,7 & 223,7 \\ 
\hline 
\end{tabular} 

\chapter{Testresultaten Iterated Local Search}
\section{9 * 9 puzzels (n = 3)}
\subsection{Puzzel 1 (test\_n3\_1\_ci)}

\subsection{Puzzel 2 (test\_n3\_2\_menneske)}
\subsection{Puzzel 3 (test\_n3\_3\_menneske)}
\subsection{Puzzel 4 (test\_n3\_4\_menneske)}
\subsection{Puzzel 5 (test\_n3\_5\_menneske)}
\section{16 * 16 puzzels (n = 4)}
\subsection{Puzzel 1 (test\_n4\_1\_ci)}

\subsection{Puzzel 2 (test\_n4\_2\_menneske)}
\subsection{Puzzel 3 (test\_n4\_3\_menneske)}
\subsection{Puzzel 4 (test\_n4\_4\_menneske)}
\subsection{Puzzel 5 (test\_n4\_5\_menneske)}
\section{25 * 25 puzzels (n = 5)}
\subsection{Puzzel 1 (test\_n5\_1\_ci)}

\subsection{Puzzel 2 (test\_n5\_2\_menneske)}
\subsection{Puzzel 3 (test\_n5\_3\_menneske)}
\subsection{Puzzel 4 (test\_n5\_4\_menneske)}
\subsection{Puzzel 5 (test\_n5\_5\_menneske)}

\chapter{Testresultaten Constraint Satisfaction - backtracking}
\section{9 * 9 puzzels (n = 3)}
\subsection{Puzzel 1 (test\_n3\_1\_ci)}

\subsection{Puzzel 2 (test\_n3\_2\_menneske)}
\subsection{Puzzel 3 (test\_n3\_3\_menneske)}
\subsection{Puzzel 4 (test\_n3\_4\_menneske)}
\subsection{Puzzel 5 (test\_n3\_5\_menneske)}
\section{16 * 16 puzzels (n = 4)}
\subsection{Puzzel 1 (test\_n4\_1\_ci)}

\subsection{Puzzel 2 (test\_n4\_2\_menneske)}
\subsection{Puzzel 3 (test\_n4\_3\_menneske)}
\subsection{Puzzel 4 (test\_n4\_4\_menneske)}
\subsection{Puzzel 5 (test\_n4\_5\_menneske)}
\section{25 * 25 puzzels (n = 5)}
\subsection{Puzzel 1 (test\_n5\_1\_ci)}

\subsection{Puzzel 2 (test\_n5\_2\_menneske)}
\subsection{Puzzel 3 (test\_n5\_3\_menneske)}
\subsection{Puzzel 4 (test\_n5\_4\_menneske)}
\subsection{Puzzel 5 (test\_n5\_5\_menneske)}

\chapter{Testresultaten Constraint Satisfaction - forward checking}
\section{9 * 9 puzzels (n = 3)}
\subsection{Puzzel 1 (test\_n3\_1\_ci)}

\subsection{Puzzel 2 (test\_n3\_2\_menneske)}
\subsection{Puzzel 3 (test\_n3\_3\_menneske)}
\subsection{Puzzel 4 (test\_n3\_4\_menneske)}
\subsection{Puzzel 5 (test\_n3\_5\_menneske)}
\section{16 * 16 puzzels (n = 4)}
\subsection{Puzzel 1 (test\_n4\_1\_ci)}

\subsection{Puzzel 2 (test\_n4\_2\_menneske)}
\subsection{Puzzel 3 (test\_n4\_3\_menneske)}
\subsection{Puzzel 4 (test\_n4\_4\_menneske)}
\subsection{Puzzel 5 (test\_n4\_5\_menneske)}
\section{25 * 25 puzzels (n = 5)}
\subsection{Puzzel 1 (test\_n5\_1\_ci)}

\subsection{Puzzel 2 (test\_n5\_2\_menneske)}
\subsection{Puzzel 3 (test\_n5\_3\_menneske)}
\subsection{Puzzel 4 (test\_n5\_4\_menneske)}
\subsection{Puzzel 5 (test\_n5\_5\_menneske)}

\end{appendices}

\end{document}          
